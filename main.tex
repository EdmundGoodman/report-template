\documentclass[12pt]{article}
% Set paper geometry
\usepackage{geometry}
\geometry{a4paper,total={170mm,257mm}}

% Set font
\usepackage{fontspec}
\setmainfont{Linux Libertine O}[Ligatures={Common}]
\setmonofont{Fira Code}[Contextuals=Alternate,Scale=MatchLowercase]

% Include useful packages
\usepackage{float}
\usepackage{graphicx}
\usepackage{appendix}
\usepackage{placeins}
\usepackage{subcaption}
\usepackage{amsmath}

% Configure hyperlinks
\usepackage{url}
\usepackage{hyperref}
\hypersetup{
	colorlinks=true,
	linkcolor=black,
	urlcolor=black,
	citecolor=black
}

% Style paragraphs
\usepackage{parskip}
\setlength{\parindent}{0pt}
\usepackage[normalem]{ulem}
\useunder{\uline}{\ul}{}

% Style captions
\usepackage{caption}
\captionsetup{labelfont=bf,width=0.8 \textwidth}

% Set up code highlight
\usepackage{minted}
\setminted{escapeinside=||,linenos,breaklines,autogobble}
%% Support listings which wrap over pages
\newenvironment{longlisting}{\captionsetup{type=listing}}{}

% Configure bibliography
\usepackage[numbib]{tocbibind}
\usepackage[citestyle=ieee]{biblatex}
\addbibresource{references.bib}
%% Avoid overflowing URLs (https://stackoverflow.com/a/43593557)
\setcounter{biburllcpenalty}{7000}
\setcounter{biburlucpenalty}{8000}

% Configure custom title section
% Abstracts can optional be defined as with titles/authors through the `\abstracttext{Text...}` command
\makeatletter
\newcommand{\@abstracttext}[0]{}
\newcommand{\abstracttext}[1]{\renewcommand{\@abstracttext}{#1}}
\newcommand{\@abstractkeywords}[0]{}
\newcommand{\abstractkeywords}[1]{\renewcommand{\@abstractkeywords}{#1}}
\newcommand{\@wordcount}[0]{}
\newcommand{\wordcount}[1]{\renewcommand{\@wordcount}{#1}}

\renewcommand\maketitle{
{\raggedright
\begin{center}
    {\Huge\textbf{\@title}} \\[0.5em]
    {\Large{\@author}} \\[0.5em]
    {\normalsize\textit{\@date}}
\end{center}
}
\ifdefempty{\@abstracttext}{}{
    %% Simple abstract
    % \section*{Abstract}
    % \@abstracttext
    % \ifdefempty{\@abstractkeywords}{}{
    %     \\\\
    %     \textbf{Keywords: }\@abstractkeywords
    % }
    %% Abstract environment without heading
    \renewcommand{\abstractname}{\vspace{-\baselineskip}}
    \begin{abstract}
        \noindent\ignorespaces\@abstracttext
        \ifdefempty{\@abstractkeywords}{}{
            \\\\
            \noindent\ignorespaces\textbf{Keywords: }\@abstractkeywords
        }
        \ifdefempty{\@wordcount}{}{
            \\\\
            \noindent\ignorespaces\textbf{Word count: }\@wordcount
        }
    \end{abstract}
}
}
\makeatother



\usepackage{amsmath}

\title{Essay title}
\author{Edmund Goodman}
\date{\today}
\abstracttext{This is an optional abstract...}
\abstractkeywords{Optional, Keywords, Here}
\wordcount{158}


\begin{document}
\maketitle

\section{Introduction}
\label{sec:introduction}

In 1968, Donald Knuth published the first volume of ``The Art of Computer Programming'' \cite{knuth1997art}, which would later be one of the first documents to be typeset in \TeX.

It's derivative, \LaTeX, can easily typeset the majority of what a computer scientist might like to write whilst avoiding the scourge of Microsoft. For example, syntax-highlighted code shown in Listing \ref{listing:haskell-lasagne-monoid}, individual images shown in Figure \ref{fig:example-image-a}, side-by-side images shown in Figure \ref{fig:side-by-side}, single and multi-line equations shown in Equations \ref{eq:amdahls-law} and \ref{eq:gustafsons-law} respectively, and references to sections such as §\ref{sec:snippets}.

\section{Snippets}
\label{sec:snippets}

\begin{listing}[H]
    \begin{minted}{haskell}
        newtype Lasagne = Lasagne Int
            deriving (Show, Num)

        -- The stacking operating can be considered integer
        -- addition of the number of layers
        instance Semigroup Lasagne where
            (<>) = (+)

        -- The identity element is the empty (zero-layer) Lasagne
        instance Monoid Lasagne where
            mempty = Lasagne 0

        -- Stacking 5 and 6 layers gives 11 layers:
        --
        -- ghci> Lasagne 5 <> Lasagne 6
        -- Lasagne 11
    \end{minted}
    \caption{A Haskell implementation of the Lasagne monoid, which is not an endofunctor.}
    \label{listing:haskell-lasagne-monoid}
\end{listing}

\begin{figure}[H]
    \centering
    \includegraphics[width=0.5\textwidth]{example-image-a}
    \caption{An example image.}
    \label{fig:example-image-a}
\end{figure}

\begin{figure}[H]
     \centering
     \begin{subfigure}[b]{0.45\textwidth}
         \centering
         \includegraphics[width=\textwidth]{example-image-a}
         \caption{Example image A.}
         \label{fig:example-image-a-side}
     \end{subfigure}
     \hfill
     \begin{subfigure}[b]{0.45\textwidth}
         \centering
         \includegraphics[width=\textwidth]{example-image-b}
         \caption{Example image A.}
         \label{fig:example-image-b-side}
     \end{subfigure}
     \caption{Two example images.}
     \label{fig:side-by-side}
\end{figure}

\begin{equation}
    S = \frac{1}{f + \frac{1-f}{P}}
    \label{eq:amdahls-law}
\end{equation}

\begin{align}
    S &= s + p \times N \\
      &= N + (1 - N) \times s \label{eq:gustafsons-law}
\end{align}

\printbibliography[heading=bibnumbered]
\appendix

\end{document}
